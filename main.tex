\documentclass[oneside]{book}
\usepackage{xcolor}
\usepackage{graphicx}
\usepackage{subcaption}
\usepackage[inline,shortlabels]{enumitem}
\usepackage{multicol}
\usepackage{multirow}
\usepackage{booktabs,array,adjustbox}

\usepackage[margin=2.5cm]{geometry}

\graphicspath{{./picture/}}

\newcommand{\exercisename}{Latihan}
\newcommand{\solutionname}{Solusi}

\definecolor{main}{RGB}{0,120,2}

%% Exercise with counter
\newcounter{exer}[chapter]
\setcounter{exer}{0}
\renewcommand{\theexer}{\thechapter.\arabic{exer}}
\newenvironment{exercise}[1][]{
  \refstepcounter{exer}
  \par\noindent\textbf{\color{main}{\exercisename} \theexer #1 }\rmfamily}{
  \par\ignorespacesafterend}

\newenvironment{solution}{\par\noindent\textbf{\color{main}\solutionname} \em}{\par}

\begin{document}

\chapter{Digital Transmission}

% Johanes Wilian Ang 1-4
\begin{exercise} % NO.1

Hitung nilai kecepatan sinyal untuk setiap kasus pada Gambar 4.2 jika kecepatan data 1 Mbps dan c = 1/2.
\end{exercise}

\begin{solution}

rumus=c×N×(1/r)untuk setiap kasus.Weletc=1/2. sebuah. 
\begin{enumerate}[a]
  \item r=1 → s=(1/2)×(1Mbps)×1/1 = 500kbaud
  \item r=1/2 → s=(1/2)×(1Mbps)×1/(1/2) = 1Mbaud
  \item r=2 → s=(1/2)×(1Mbps)×1/2 = 250Kbaud
  \item r=4/3 → s=(1/2)×(1Mbps)×1/(4/3) = 375Kbaud
\end{enumerate}
\end{solution}

\vspace{12pt}

\begin{exercise} % No.2

Dalam transmisi digital, waktu pengirim 0,2 persen lebih cepat dari waktu penerima. Berapa bit per detik yang dikirim pengirim jika kecepatan data 1 Mbps?
\end{exercise}

\begin{solution}
  
rumus jumlah bit  (0,2/100) × (1 Mbps) = 2000 bit
\end{solution}

\vspace{12pt}

\begin{exercise} %No.3
  
Gambarlah grafik skema NRZ-L menggunakan masing-masing aliran data berikut, dengan asumsi bahwa level sinyal terakhir adalah positif. Dari grafik, tebak bandwidth untuk skema ini menggunakan jumlah rata-rata perubahan level sinyal. Bandingkan perhitungan Anda dengan entri yang sesuai pada Tabel 4.1.
\begin{enumerate}[a]
  \item 00000000
  \item 11111111
  \item 01010101
  \item 00110011
\end{enumerate}
\end{exercise}

\begin{solution}

  Bandwidth dengan perbandingan (3/8)N yang berada dalam kisaran (B = 0 sampai N) untuk skema NRZ-L.

    \includegraphics[width=0.9\textwidth]{gambar_solusi_No.3.PNG}

   \centering Gambar 1.1 perbadingan NRZ-L untuk No.3


\end{solution}

\vspace{12pt}


\begin{exercise} % No.4
  
Ulangi Latihan 15 untuk skema NRZ-I
\end{exercise}

\begin{solution}

  Bandwidth dengan perbadingan (4.25/8)N yang berada dalam kisaran pada (B = 0 hingga N) untuk skema NRZ-I.
\vspace{12pt}

  \includegraphics[width=0.9\textwidth]{gambar_solusi_No.4.PNG}
  \centering Gambar 1.2 Perbandingan NRZ-I untuk No.4
\end{solution}

\vspace{12pt}

% Erwin Erikson 5-8
\begin{exercise}

  Gambarlah grafik skema Manchester menggunakan masing-masing aliran data berikut, dengan asumsi bahwa level sinyal terakhir adalah positif. Dari grafik, tebak bandwidth untuk skema ini menggunakan jumlah rata-rata perubahan level sinyal.
  \begin{enumerate}[a)]
    \item 00000000
    \item 11111111
    \item 01010101
    \item 00110011
  \end{enumerate}

  Bandingkan tebakan Anda dengan entri yang sesuai pada Tabel 1.1.

  \begin{table}[htbp]
    \begin{center}
      \centerline{Tabel 1.1: Ringkasan skema pengkodean baris}
      \begin{tabular}{|l|l|c|p{6cm}|}
        \cline{1-4}
        \multirow{2}{*}{Kategori}&\multirow{2}{*}{Skema}&\multicolumn{1}{c|}{Bandwidth}&\multirow{2}{*}{\centerline{Karakteristik}} \\
        & &\multicolumn{1}{c|}{(rata-rata)}& \\
        \cline{1-4}
        Unipolar&NRZ&B=N/2&Mahal, tidak ada sinkronisasi otomatis jika panjang Os atau Is, DC \\
        \cline{1-4}
        \multirow{3}{*}{Unipolar}&NRZ-L&B=N/2&Tidak ada sinkronisasi sendiri jika Os panjang atau 1s, DC \\
        \cline{2-4}
        &NRZ-I&B=N/2&Tidak ada sinkronisasi otomatis selama S, DC \\
        \cline{2-4}
        &Biphase&B=N&Sinkronisasi diri, tidak ada DC, bandwidth tinggi \\
        \cline{1-4}
        Bipolar&AMI&B=N/2&Tidak ada sinkronisasi otomatis untuk OS lama, DC \\
        \cline{1-4}
        \multirow{3}{*}{Multilevel}&2BIQ&B=N/4&Tidak ada sinkronisasi sendiri untuk bit ganda yang sama panjang \\
        \cline{2-4}
        &8B6T&B=3N/4&Sinkronisasi diri, tidak ada DC \\
        \cline{2-4}
        &4D-PAM5&B=N/8&Sinkronisasi diri, tidak ada DC \\
        \cline{1-4}
        Multiline&MLT-3&B=N/3&Tidak ada sinkronisasi otomatis untuk Os yang lama \\
        \cline{1-4}
      \end{tabular}
    \end{center}\vspace*{6px}
  \end{table}\vspace*{8px}

\end{exercise}

\begin{solution}

  Gambar grafik skema Manchester dapat dilihat pada Gambar 1.1.

\begin{figure}[ht]
  \centering
  \includegraphics[width=0.9\textwidth]{gbr_solusi_1_5.jpg}
  \centerline{Gambar 1.1: Gambar grafik skema Manchester}\vspace*{2px}
\end{figure}

Bandwidth sebanding dengan (12,5 / 8) N yang berada dalam kisaran pada Tabel 1.1

(B = N hingga B = 2N) untuk skema Manchester.

\end{solution}

\vspace{12pt}

\begin{exercise}

  Gambarlah grafik skema diferensial Manchester menggunakan masing-masing aliran data berikut, dengan asumsi bahwa level sinyal terakhir adalah positif. Dari grafik, tebak bandwidth untuk skema ini menggunakan jumlah rata-rata perubahan level sinyal.
  \begin{enumerate}[a)]
    \item 00000000
    \item 11111111
    \item 01010101
    \item 00110011
  \end{enumerate}

  Bandingkan tebakan Anda dengan entri yang sesuai pada Tabel 1.1.\vspace*{8px}

\end{exercise}

\begin{solution}

  Gambar grafik skema diferensial Manchester dapat dilihat pada Gambar 1.2.

\begin{figure}[ht]
  \centering
  \includegraphics[width=0.9\textwidth]{gbr_solusi_1_6.jpg}
  \centerline{Gambar 1.2: Gambar grafik skema diferensial Manchester}\vspace*{2px}
\end{figure}

Bandwidth sebanding dengan (12/8) N yang berada dalam kisaran pada Tabel 1.1

(B = N ke 2N) untuk skema diferensial Manchester.

\end{solution}

\vspace{12pt}

\begin{exercise}

  Gambarlah grafik skema 2B1Q menggunakan masing-masing aliran data berikut, dengan asumsi bahwa level sinyal terakhir adalah positif. Dari grafik, tebak bandwidth untuk skema ini menggunakan jumlah rata-rata perubahan level sinyal.
  \begin{enumerate}[a)]
    \item 0000000000000000
    \item 1111111111111111
    \item 0101010101010101
    \item 0011001100110011
  \end{enumerate}

  Bandingkan tebakan Anda dengan entri yang sesuai pada Tabel 1.1.\vspace*{8px}

\end{exercise}

\begin{solution}

  Gambar grafik skema 2B1Q dapat dilihat pada Gambar 1.3.

\begin{figure}[ht]
  \centering
  \includegraphics[width=0.9\textwidth]{gbr_solusi_1_7.jpg}
  \centerline{Gambar 1.3: Gambar grafik skema 2B1Q}\vspace*{2px}
\end{figure}

Bandwidth sebanding dengan (5.25 / 16) N yang berada di dalam range pada Tabel 1.1

(B = 0 hingga N/2) untuk 2B/1Q.

\end{solution}

\vspace{12pt}

\begin{exercise}

  Gambarlah grafik skema MLT-3 menggunakan masing-masing aliran data berikut, dengan asumsi bahwa level sinyal terakhir adalah positif. Dari grafik, tebak bandwidth untuk skema ini menggunakan jumlah rata-rata perubahan level sinyal.
  \begin{enumerate}[a)]
    \item 00000000
    \item 11111111
    \item 01010101
    \item 00011000
  \end{enumerate}

  Bandingkan tebakan Anda dengan entri yang sesuai pada Tabel 1.1.\vspace*{8px}

\end{exercise}

\begin{solution}

  Gambar grafik skema MLT-3 dapat dilihat pada Gambar 1.4.

\begin{figure}[ht]
  \centering
  \includegraphics[width=0.9\textwidth]{gbr_solusi_1_7.jpg}
  \centerline{Gambar 1.4: Gambar grafik skema MLT-3}\vspace*{2px}
\end{figure}

Bandwidth sebanding dengan (5.25/8) $\times$ N yang berada di dalam kisaran pada Tabel 1.1

(B = 0 hingga N/2) untuk MLT-3.

\end{solution}

\vspace{12pt}

% Johnny 9-12 Test
\begin{exercise}
  Contoh soal 9
\end{exercise}

\begin{solution}
  Contoh solusi
\end{solution}

\vspace{12pt}

\begin{exercise}
  Contoh soal
\end{exercise}

\begin{solution}
  Contoh solusi
\end{solution}

\vspace{12pt}

\begin{exercise}
  Contoh soal
\end{exercise}

\begin{solution}
  Contoh solusi
\end{solution}

\vspace{12pt}

\begin{exercise}
  Contoh soal
\end{exercise}

\begin{solution}
  Contoh solusi
\end{solution}

\vspace{12pt}

% Riani Artika 13-16
\begin{exercise}
  Contoh soal 13
\end{exercise}

\begin{solution}
  Contoh solusi
\end{solution}

\vspace{12pt}

\begin{exercise}
  Contoh soal
\end{exercise}

\begin{solution}
  Contoh solusi
\end{solution}

\vspace{12pt}

\begin{exercise}
  Contoh soal
\end{exercise}

\begin{solution}
  Contoh solusi
\end{solution}

\vspace{12pt}

\begin{exercise}
  Contoh soal
\end{exercise}

\begin{solution}
  Contoh solusi
\end{solution}

\vspace{12pt}

% Muhammad Al Imron 17-20
\begin{exercise}

  Berapa kecepatan maksimum data dari sebuah saluran dengan bandwidth 200 KHz jika kita menggunakan 4 level sinyal digital ?
\end{exercise}

\begin{solution}

  Kecepatan maksimum data rate dapat di hitung menggunakan :
  Nmax = 2 $\times$ B $\times$ nb = 2 $\times$ 200 KHz $\times$ log24 = 800 kbps
\end{solution}

\vspace{12pt}

\begin{exercise}

  Sebuah sinyal analog memiliki bandwidth 20 Khz. Jika kita mengambil sampel sinyal ini dan mengirimkannya melalui saluran 30 Kbps, berapakah SNRdB dari sinyal tersebut ?
\end{exercise}

\begin{solution}

  Pertama kita menghitung sampling rate (fs) dan angka bit dari setiap sample (nb).
  \vspace{6pt}

  fmax = 0 + 4 = 4 KHz → fs = 2 $\times$ 4 = 8000 sample/s
  \vspace{6pt}

  Kemudian kita menghitung angka bit per-sample.
  \vspace{6pt}

  → nb = 30000 / 8000 = 3.75
  \vspace{6pt}

  Kita juga menggunakan integer nb = 4.
  \vspace{6pt}

  Nilai SNRdB = 6.02 $\times$ nb + 1.72 = 25.8
  \vspace{6pt}

\end{solution}

\vspace{12pt}

\begin{exercise}
  
  Kami memiliki saluran baseband dengan bandwidth I-MHz. Berapa kecepatan data untuk saluran ini jika kita menggunakan salah satu skema pengkodean baris berikut?
  \begin{enumerate}[a)]
    \item NRZ-L
    \item Manchester
    \item MLT-3
    \item 2B1Q
  \end{enumerate}
\end{exercise}

\begin{solution}
  
  Kita bisa menghitung data rate dari setiap scheme dengan :
  \begin{enumerate}[a)]
    \item NRZ → N = 2 $\times$ B = 2 $\times$ 1 MHz = 2 Mbps
    \item Manchester → N = 1 $\times$ B = 1 $\times$ 1 MHz = 1 Mbps
    \item MLT-3 → N = 3 $\times$ B = 3 $\times$ 1 MHz = 3 Mbps
    \item 2B1Q → N = 4 $\times$ B = 4 $\times$ 1 MHz = 4 Mbps
  \end{enumerate}
\end{solution}

\vspace{12pt}

\begin{exercise}
  
  Kita akan mengirimkan 1000 karakter dengan setiap karakter nya di encode dengan 8 bit.
  \begin{enumerate}[a)]
    \item Tentukan angka dari yang di bit yang dikirimkan untuk sychronous transmission !
    \item Tentukan angka dari yang di bit yang dikirimkan untuk asychronous transmission !
    \item Tentukan persentase redudansi dari setiap case ! 
  \end{enumerate}
\end{exercise}

\begin{solution}
  \begin{enumerate}[a)]
    \item Untuk sychronous transmission, kita memiliki 1000 × 8 = 8000 bits.
    \item Untuk asychronous transmission, kita memiliki 1000 × 10 = 10000 bits. Catatan : kita berasumsi hanya berhenti 1 bit dan memulai 1 bit. Beberapa sistem menggunakan bit lebih.
    \item Untuk case a, redudansi nya 0\%. Untuk case b, kita mengirim extra 2000 untuk setiap 8000 yang bit butuhkan. redudansi nya 25\%.
  \end{enumerate}
\end{solution}

\chapter{Analog Transmission}

% Rian Sanjaya Nadeak 1-4
\begin{exercise}
  Calculate the baud rate for the given bit rate and type of modulation.
  \begin{itemize}
    \item[a.] 2000 bps, FSK
    \item[b.] 4000 bps, ASK
  \end{itemize}
\end{exercise}

\begin{solution}
  We use the formula $S = (1/r) \times N$, but first we need to calculate the value of r for each case.
  \begin{itemize}
    \item[a.] $r = log_22 = 1 \quad \rightarrow \quad S = (1/1) \times (2000 \textnormal{ bps}) = 2000 \textnormal{ baud}$
    \item[b.] 
  \end{itemize}
\end{solution}

\vspace{12pt}

\begin{exercise}
  Contoh soal
\end{exercise}

\begin{solution}
  Contoh solusi
\end{solution}

\vspace{12pt}

\begin{exercise}
  Contoh soal
\end{exercise}

\begin{solution}
  Contoh solusi
\end{solution}

\vspace{12pt}

\begin{exercise}
  Contoh soal
\end{exercise}

\begin{solution}
  Contoh solusi
\end{solution}

\vspace{12pt}

% R.M. Fikri Ihsan Kurniawan 5-8
\begin{exercise}
  Gambarlah diagram konstelasi untuk kasus-kasus berikut. Temukan amplitudo puncak nilai untuk setiap kasus dan tentukan jenis modulasi (ASK, FSK, PSK, atau QAM).Angka-angka dalam tanda kurung menentukan nilai I dan Q masing-masing.
  \begin{itemize}
    \item[a.] Dua titik di (2, 0) dan (3, 0).
    \item[b.] Dua titik di (3, 0) dan (-3, 0).
    \item[c.] Empat poin di (2, 2), (-2, 2), (-2, -2), dan (2, -2).
    \item[d.] Dua titik di (0 , 2) dan (0, -2).
  \end{itemize}
    
\end{exercise}

\begin{solution}
  \begin{figure}[ht]
  \centering
  \includegraphics[width=0.9\textwidth]{picture/figure-5.2.png}
  \centerline{Solution of Latihan 2.5}\vspace*{2px}
\end{figure}
\begin{itemize}
    \item[a.] Ada dua amplitudo puncak keduanya dengan fase yang sama (0 derajat). Nilai amplitudo puncak adalah $A_1$ = 2 (jarak antara titik pertama dan titik asal) dan $A_2$ = 3 (jarak antara titik kedua dan titik asal). 
    \item[b.] Hanya ada satu amplitudo puncak (3). Jarak antara masing-masing titik dan asalnya adalah 3. Namun, kami memiliki dua fase, 0 dan 180 derajat.
    \item[c.] Ini dapat berupa QPSK (satu amplitudo, empat fase) atau 4-QAM (satu amplitudo dan empat fase). Amplitudo adalah jarak antara titik dan asal, yaitu $ (2^2 + 2^2)^\frac{1}{2} $ = 2,83.
    \item[d.] Ini juga BPSK. Amplitudo puncaknya adalah 2, tetapi kali ini fasenya adalah 90 dan 270 derajat.
  \end{itemize} 
\end{solution}

\vspace{12pt}

\begin{exercise}
  Berapa banyak bit per baud yang dapat kita kirim dalam setiap kasus berikut jika sinyal: rasi bintang memiliki salah satu dari jumlah titik berikut?
  \begin{itemize}
    \item[a.] 2
    \item[b.] 4
    \item[c.] 16
    \item[d.] 1024
  \end{itemize}
\end{exercise}

\begin{solution}
  Jumlah titik menentukan jumlah level, L. Jumlah bit per baud
adalah nilai r. Oleh karena itu, kami menggunakan rumus r = log$_2$L untuk setiap kasus.
\begin{itemize}
    \item[a.] $ log_2 $2 = 1
    \item[b.] $ log_2 $4 = 2
    \item[c.] $ log_2 $16 = 4
    \item[d.] $ log_2 $1024 = 10
 \end{itemize}
\end{solution}

\vspace{12pt}

\begin{exercise}
 Berapa bandwidth yang dibutuhkan untuk kasus berikut jika kita perlu mengirim 4000 bps? Misalkan d = 1.
  \begin{itemize}
    \item[a.] ASK
    \item[b.] QPSK
    \item[c.] 16-QAM
    \item[d.] 64-QAM
  \end{itemize}
\end{exercise}

\begin{solution}
  Kami menggunakan rumus B = (1 + d) $\times$ (1/r) $\times$ N, tetapi pertama-tama kita perlu menghitung nilai r untuk setiap kasus.
  \begin{itemize}
    \item[a.] r = 1 → B = (1 + 1) $\times$ (1/1) $\times$ (4000 bps) = 8000 Hz
    \item[b.] r = 1 → B = (1 + 1) $\times$ (1/1) $\times$ (4000 bps) + 4 KHz = 8000 Hz
    \item[c.] r = 2 → B = (1 + 1) $\times$ (1/2) $\times$ (4000 bps) = 2000 Hz
    \item[d.] r = 4 → B = (1 + 1) $\times$ (1/4) $\times$ (4000 bps) = 1000 Hz
 \end{itemize}
\end{solution}

\vspace{12pt}

\begin{exercise}
  Saluran telepon memiliki bandwidth 4 KHz. Berapa jumlah bit maksimum yang kami? dapat mengirim menggunakan masing-masing teknik berikut? Misalkan d = O
   \begin{itemize}
    \item[a.] ASK
    \item[b.] QPSK 
    \item[c.] 16-QAM
    \item[d.] 64-QAM
    \end{itemize}
\end{exercise}

\begin{solution}
  Kami menggunakan rumus N = [1/(1 + d)] $\times$ r $\times$ B, tetapi pertama-tama kita perlu menghitung nilai r untuk setiap kasus.
  \begin{itemize}
    \item[a.] r = log$_2$2 = 1 → N= [1/(1 + 0)] $\times$ 1 $\times$ (4 KHz) = 4 kbps
    \item[b.] r = log$_2$4=2 → N = [1/(1 + 0)] $\times$ 2 $\times$ (4 KHz) = 8 kbps
    \item[c.] r = log$_2$16= 4 → N = [1/(1 + 0)] $\times$ 4 $\times$ (4 KHz) = 16 kbps
    \item[d.] r = log$_2$64= 6 → N = [1/(1 + 0)] $\times$ 6 $\times$ (4 KHz) = 24 kbps
    \end{itemize}
\end{solution}

\vspace{12pt}

% Farhan Ghulam Hadi Saputra 9-12
\begin{exercise}
  Sebuah perusahaan memiliki media dengan bandwidth I-MHz (lowpass). Korporasi perlu membuat 10 saluran independen terpisah yang masing-masing mampu mengirim setidaknya 10Mbps. Perusahaan telah memutuskan untuk menggunakan teknologi QAM. Berapa minimal? jumlah bit per baud untuk setiap saluran? Berapa jumlah poin dalam diagram konstelasi untuk setiap saluran? Misalkan d = O.
\end{exercise}

\begin{solution}
  Pertama, kami menghitung bandwidth untuk setiap saluran = (1 MHz) / 10 = 100 KHz. Kita kemudian cari nilai r untuk setiap saluran: B = (1 + d) × (1/r) × (N) → r = N / B → r = (1 Mbps/100 KHz) = 10 Kita kemudian dapat menghitung jumlah level: L = 2r = 210 = 1024. Ini berarti bahwa bahwa kita memerlukan teknik 1024-QAM untuk mencapai kecepatan data ini.
\end{solution}

\vspace{12pt}

\begin{exercise}
  Sebuah perusahaan kabel menggunakan salah satu saluran TV kabel (dengan bandwidth 6 MHz) untuk menyediakan komunikasi digital untuk setiap penduduk. Berapa kecepatan data yang tersedia? untuk setiap penduduk apakah perusahaan menggunakan teknik 64-QAM?
\end{exercise}

\begin{solution}
  Kita dapat menggunakan rumus: N = [1/(1 + d)] × r × B = 1 × 6 × 6 MHz = 36 Mbps
\end{solution}

\vspace{12pt}

\begin{exercise}
  Temukan bandwidth untuk situasi berikut jika kita perlu memodulasi 5-KHz suara.
  \begin{itemize}
    \item[a.] AM
    \item[b.] PM (set $ \beta$=5)
    \item[c.] PM (set $ \beta$=1)
    \end{itemize}
\end{exercise}

\begin{solution}
 \begin{itemize}
    \item[a.] $ B_{AM} $ = 2 × B = 2 × 5 = 10 KHz
    \item[b.] $ B_{FM} $ = 2 × (1 + $\beta$) × B = 2 × (1 + 5) × 5 = 60 KHz
    \item[c.] $ B_{PM} $ = 2 × (1 + $\beta$) × B = 2 × (1 + 1) × 5 = 20 KHz
    \end{itemize}
\end{solution}

\vspace{12pt}

\begin{exercise}
  Temukan jumlah total saluran di pita yang sesuai yang dialokasikan oleh FCC.
\begin{itemize}
    \item[a.] AM
    \item[b.] FM
    \end{itemize}
\end{exercise}

\begin{solution}
  Kami menghitung jumlah saluran, bukan jumlah stasiun yang hidup berdampingan
  \begin{itemize}
    \item[a.] n = (1700 - 530) KHz / 10 KHz = 117
    \item[b.] n = (108 - 88) MHz / 200 KHz = 100
    \end{itemize}
\end{solution}

\chapter{Bandwidth Utilization: Multiplexing and Spreading}

% Hani Khairiyah 1-4
\begin{exercise}
  Asumsikan bahwa saluran suara menempati bandwidth 4 kHz. Kita perlu multipleks 10 saluran suara dengan band penjaga 500 Hz menggunakan FDM. Hitung yang dibutuhkan bandwith?
\end{exercise}

\begin{solution}
  Untuk multipleks 10 saluran suara, kita membutuhkan sembilan band penjaga. Bandwidth yang dibutuhkan kemudian B = (4 KHz) * 10 + (500 Hz) * 9 = 44,5 KHz
\end{solution}

\vspace{12pt}

\begin{exercise}
  Kita perlu mengirimkan 100 saluran suara digital menggunakan saluran pass-band dari 20 KHz. Berapa rasio bit/Hz jika kita tidak menggunakan guard band?
\end{exercise}

\begin{solution}
  Bandwidth yang dialokasikan untuk setiap saluran suara adalah 20 KHz / 100 = 200 Hz. Seperti yang kita lihat di bab sebelumnya, setiap saluran suara digital memiliki kecepatan data 64 Kbps (8000 sampel * 8 bit/sampel). Ini berarti bahwa teknik modulasi kami menggunakan 64.000/200 = 320 bit/Hz.
\end{solution}

\vspace{12pt}

\begin{exercise}
  Dalam hierarki analog Gambar 6.9, temukan overhead bandwidth ekstra untuk guard
band atau kontrol di setiap level hierarki grup, supergrup, grup master, dan
kelompok jumbo.
\end{exercise}

\begin{solution}
  a. Tingkat grup: overhead = 48 KHz (12 * 4 KHz) = 0 Hz
  b. Tingkat supergrup: overhead = 240 KHz (5 * 48 KHz) = 0 Hz
  c. Grup master: overhead = 2520 KHz (10 * 240 KHz) = 120 KHz
  d. Grup Jumbo: overhead = 16,984 MHz (6 * 2,52 MHz) = 1,864 MHz.
\end{solution}

\vspace{12pt}

\begin{exercise}
  Kita perlu menggunakan TDM sinkron dan menggabungkan 20 sumber digital, masing-masing 100 Kbps. Setiap slot keluaran membawa 1 bit dari setiap sumber digital, tetapi satu bit tambahan ditambahkan ke setiap frame untuk sinkronisasi. Jawab pertanyaan berikut
  a. Berapa ukuran bingkai keluaran dalam bit?
  b. Berapa kecepatan bingkai keluaran?
  c. Berapa durasi frame output?
  d. Berapa kecepatan data keluaran?
  e. Berapa efisiensi sistem (rasio bit yang berguna dengan total bit).
\end{exercise}

\begin{solution}
  a. Setiap bingkai keluaran membawa 1 bit dari setiap sumber ditambah satu bit tambahan untuk sinkronisasi. Ukuran bingkai = 20 * 1 + 1 = 21 bit.
  b. Setiap frame membawa 1 bit dari setiap sumber. Kecepatan bingkai = 100.000 bingkai/dtk.\
  c. Durasi bingkai = 1 /(kecepatan bingkai) = 1/100.000 = 10 s
  d. Kecepatan data = (100.000 bingkai/dtk) * (21 bit/bingkai) = 2,1 Mbps
  e. Dalam setiap frame 20 bit dari 21 berguna. Efisiensi = 20/21= 95%
\end{solution}

\vspace{12pt}

% Kevin Antony Kasamilale 5-8
\begin{exercise}
  Contoh soal 5 : Kita perlu menggunakan TDM sinkron dan menggabungkan 20 sumber digital, masing-masing 100 Kbps.
   Setiap slot keluaran membawa 1 bit dari setiap sumber digital, tetapi satu bit tambahan ditambahkan ke setiap frame untuk sinkronisasi
  \end{exercise}
  
  \begin{itemize}
    \item Cara menghitung Berapa ukuran bingkai keluaran dalam bit?
    \item Cara menghitung Berapa kecepatan bingkai keluaran?
    \item Cara menghitung Berapa durasi frame output?
    \item Cara menghitung Berapa kecepatan data keluaran?
    \item Cara menghitung Berapa efisiensi sistem (rasio bit yang berguna dengan total bit)?
    setiap slot keluaran membawa 2 bit dari setiap sumber
  \end{itemize}

\begin{solution}
  Contoh solusi

\begin{itemize}
  \item Setiap bingkai keluaran membawa 2 bit dari setiap sumber ditambah satu bit tambahan untuk sinkronisasi. Ukuran bingkai = 20 × 2 + 1 = 41 bit.
  \item Setiap frame membawa 2 bit dari setiap sumber. Kecepatan bingkai = 100.000/2 = 50.000 bingkai/dtk.
  \item Durasi bingkai = 1 /(kecepatan bingkai) = 1 /50.000 = 20 s.
  \item Kecepatan data = (50.000 frame/s) × (41 bit/frame) = 2,05 Mbps. Kecepatan data keluaran di sini sedikit kurang dari yang ada di Latihan 16.
  \item Dalam setiap frame 40 bit dari 41 berguna. Efisiensi = 40/41= 97,5%. Efisiensi lebih baik daripada yang ada di Latihan 16.
\end{itemize}	

\end{solution}

\vspace{12pt}

\begin{exercise}
  Contoh soal 6 : Kami memiliki 14 sumber, masing-masing membuat 500 karakter 8-bit per detik. 
  Karena hanya beberapa dari sumber-sumber ini yang aktif setiap saat, kami menggunakan TDM statistik untuk menggabungkan sumber-sumber ini menggunakan penyisipan karakter. Setiap frame membawa 6 slot sekaligus, tetapi kita perlu menambahkan alamat 4-bit ke setiap slot. Jawab pertanyaan berikut:
\end{exercise}
  \begin{itemize}
      \item  Berapa ukuran bingkai keluaran dalam bit?
      \item  Berapa kecepatan bingkai keluaran?
      \item  Berapa durasi bingkai keluaran?
      \item Berapa kecepatan data keluaran?
    \end{itemize}

\begin{solution}
  
\begin{itemize}
\item [a] Ukuran bingkai = 6 × (8 + 4) = 72 bit.
\item [b] Kita dapat mengasumsikan bahwa kita hanya memiliki 6 jalur input. Setiap bingkai perlu membawa satu  karakter dari masing-masing baris ini. Ini berarti bahwa frame rate adalah 500 bingkai/dtk.
\item [c] Durasi bingkai = 1 /(kecepatan bingkai) = 1 /500 = 2 md.
\item [d] Kecepatan data = (500 frame/s) × (72 bit/frame) = 36 kbps.
\end{itemize}
\end{solution}

\vspace{12pt}

\begin{exercise}
  Contoh soal : Sepuluh sumber, enam dengan bit rate 200 kbps dan empat dengan bit rate 400 kbps, 
  akan digabungkan menggunakan TDM bertingkat tanpa bit sinkronisasi. Jawablah pertanyaan-pertanyaan berikut tentang tahap akhir dari multiplexing: 
\end{exercise}

\begin{itemize}
  \item  Berapa ukuran frame dalam bit? 
  \item Berapa frame ratenya? 
  \item Berapa durasi sebuah frame? 
  \item Berapa kecepatan datanya?
  
\end{itemize}

\begin{solution}
  \begin{itemize}
   \item 	Setiap frame keluaran membawa 1 bit dari masing-masing dari tujuh saluran 400-kbps. Ukuran bingkai = 7 × 1 = 7 bit.
    \item  Setiap frame membawa 1 bit dari setiap sumber 400-kbps. Kecepatan bingkai = 400.000 bingkai/dtk.
    \item 	Durasi bingkai = 1 /(kecepatan bingkai) = 1/400.000 = 2,5 s.
    \item Kecepatan data keluaran = (400.000 bingkai/dtk) × (7 bit/bingkai) = 2,8 Mbps. Kita juga dapat menghitung laju data keluaran sebagai jumlah laju data masukan karena tidak ada bit yang disinkronkan. Kecepatan data keluaran = 6 × 200 + 4 × 400 = 2,8 Mbps.

  \end{itemize}
\end{solution}

\vspace{12pt}

\begin{exercise}
  Contoh soal : Empat saluran, dua dengan kecepatan bit 200 kbps dan dua dengan kecepatan bit 150 kbps,
  akan dimultipleks menggunakan TDM slot ganda tanpa bit sinkronisasi. Jawab pertanyaan berikut:
\end{exercise}

\begin{itemize}
  \item   Berapa ukuran frame dalam bit?
  \item  Berapa frame ratenya?
  \item  Berapa durasi sebuah frame?
  \item  Berapa kecepatan datanya?
  
\end{itemize}

\begin{solution}
  \begin{itemize}
   \item [a] Frame membawa 4 bit dari masing-masing dua sumber pertama dan 3 bit dari masing-masing dua sumber kedua. Ukuran bingkai = 4 × 2 + 3 × 2 = 14 bit.
    \item [b] Setiap frame membawa 4 bit dari setiap sumber 200-kbps atau 3 bit dari setiap 150 kbps. Kecepatan bingkai = 200.000 / 4 = 150.000 /3 = 50.000 bingkai/dtk.
    \item [c] Durasi bingkai = 1 /(kecepatan bingkai) = 1 /50.000 = 20 s.
     \item [d] Kecepatan data keluaran = (50.000 frame/s) × (14 bit/frame) = 700 kbps. Kita juga dapat menghitung laju data keluaran sebagai jumlah laju data masukan karena tidak ada bit sinkronisasi. Kecepatan data keluaran = 2 × 200 + 2 × 150 = 700 kbps.
  
  \end{itemize}
 
\end{solution}

\vspace{12pt}

% Annisa Wijaya 9-11
\begin{exercise}
  Contoh soal 9
\end{exercise}

\begin{solution}
  Contoh solusi
\end{solution}

\vspace{12pt}

\begin{exercise}
  Contoh soal
\end{exercise}

\begin{solution}
  Contoh solusi
\end{solution}

\vspace{12pt}

\begin{exercise}
  Contoh soal
\end{exercise}

\begin{solution}
  Contoh solusi
\end{solution}

\vspace{12pt}

% Andrian Syah 12-14 test push ke github
\begin{exercise}
  Contoh soal 12 : Gambar dibawah ini menunjukkan multiplexer dalam sistem TDM sinkron. Setiap slot keluaran adalah
  panjangnya hanya 10 bit (3 bit diambil dari setiap input ditambah 1 bit framing). Apa keluarannya?
  jalur kecil? Bit tiba di multiplexer seperti yang ditunjukkan oleh panah.
\end{exercise}

\begin{figure}[h]
  \centering
  \includegraphics[width=0.9\textwidth]{soal12.png}
\end{figure}

\begin{solution}

\end{solution}

\begin{figure}[h]
  \centering
  \includegraphics[width=0.9\textwidth]{solusi12.png}
\end{figure}

\vspace{12pt}

\begin{exercise}
  Contoh soal 13 : Gambar 3.1 dibawah menunjukkan demultiplexer dalam TDM sinkron. Jika slot input adalah 16 bit
  panjang (tanpa bit framing), apa aliran bit di setiap output? Bit tiba di
  demultiplexer seperti yang ditunjukkan oleh panah.

  \begin{figure}[h]
    \centering
    \includegraphics[width=0.9\textwidth]{soal13.png}
    \caption{TDM}
  \end{figure}

\end{exercise}


\vspace{12pt}

\begin{solution}
  \begin{figure}[h]
    \centering
    \includegraphics[width=0.9\textwidth]{solusi13.png}
    \caption{solusi TDM}
  \end{figure}
\end{solution}




\begin{exercise}
  Jawab pertanyaan berikut tentang hierarki digital pada Gambar 3.3:
  \begin{itemize}
    \item[a.] Berapa overhead (jumlah bit tambahan) dalam layanan DS-l?
    \item[b.] Berapa overhead (jumlah bit tambahan) dalam layanan DS-2?
    \item[c.] Berapa overhead (jumlah bit tambahan) dalam layanan DS-3?
    \item[d.] Berapa overhead (jumlah bit tambahan) dalam layanan DS-4?
  \end{itemize}
\end{exercise}

\begin{figure}[h]
  \centering
  \includegraphics[width=0.9\textwidth]{figure6.23.png}
  \caption{Hierarki Digital}
\end{figure}

\begin{solution}
  \begin{itemize}
    \item[a.] DS-1 Overhead = 
    \begin{math}
       1.544 Mbps - (24 \times 64 kbps) = 8 kbps 
    \end{math}
   \item [b.] DS-2 Overhead =
        \begin{math}
          6.312 Mbps - (4 \times 1.544 kbps) = 136 kbps 
        \end{math}
   \item [c.] DS-3 Overhead =
        \begin{math}
          44.376 Mbps - (7 \times 6.312 kbps) = 192 kbps 
        \end{math}
   \item [d.] DS-4 Overhead =
   \begin{math}
    274.176 Mbps - (6 \times 44.376 kbps) = 7.92 Mbps 
  \end{math}
  \end{itemize}
\end{solution}

\vspace{12pt}

% Maranti Nainggolan 15-18
\begin{exercise}
  Berapa jumlah bit minimum pada urutan PN jika kita menggunakan FHSS dengan bandwidth saluran B = 4 KHz dan Bss = 100 KHz?
 \end{exercise}
 
 \begin{solution}
   Jumlah hop = 100 KHz/4 KHz = 25. Jadi kita membutuhkan log225 = 4,64 = 5 bit
 \end{solution}
 
 \vspace{12pt}
 
 \begin{exercise}
 Sistem FHSS menggunakan urutan PN 4-bit. Jika bit rate PN adalah 64 bit per detik, jawablah pertanyaan berikut: 
 \begin{itemize}
  \item[a.] Berapa jumlah total saluran yang mungkin?
  \item[b.] Berapa jumlah waktu yang diperlukan untuk menyelesaikan satu putaran penuh PN?
\end{itemize}
 \end{exercise}
 
 \begin{solution}
  \begin{itemize}
    \item[a.] 24 = 16 lompatan
    \item[b.] (64 bit/s) / 4 bit = 16 siklus
  \end{itemize}
 \end{solution}
 
 \vspace{12pt}
 
 \begin{exercise}
  Generator angka pseudorandom menggunakan rumus berikut untuk membuat deret acak: Ni11 5 (5 1 7Ni) mod 17 Di mana Ni mendefinisikan nomor acak saat ini dan Ni+1 mendefinisikan nomor acak berikutnya. Istilah mod berarti nilai sisa saat membagi (5 + 7Ni) dengan 17. Tunjukkan urutan yang dibuat oleh generator ini untuk digunakan untuk spread spectrum.
 \end{exercise}

 \vspace{12pt}
 
 \begin{solution}
 Bilangan acak adalah 11, 13, 10, 6, 12, 3, 8, 9 seperti yang dihitung di bawah ini: 
<<<<<<< HEAD

=======
 \begin{itemize}
  \item[a.] N1 = 11
  \item[b.] N2 =(5 +7 * 11) mod 17 1 = 13
  \item[c.] N3 =(5 +7 * 13) mod 17 1 = 10 
  \item[d.]N4 =(5 +7 * 10) mod 17 1 = 6
  \item[e.] N5 =(5 +7 * 6) mod 17 1 = 12 
  \item[f.]N6 =(5 +7 * 12) mod 17 1 = 3 
  \item[g.]N7 =(5 +7 * 3) mod 17 1 = 8 
  \item[h.]N8 =(5 + 7 * 8) mod 17 1 = 9
\end{itemize}
>>>>>>> e5a208f642cb6fef7bb29351946f1aafd79663ac
\end{solution}
 
 \vspace{12pt}
 
 \begin{exercise}
 Kami memiliki media digital dengan kecepatan data 10 Mbps. Berapa banyak saluran suara 64 kbps yang dapat dibawa oleh media ini jika kita menggunakan DSSS dengan urutan Barker?
 \end{exercise}
 
 \vspace{12pt}

 \begin{solution}
 Chip Barker adalah 11 bit, yang berarti meningkatkan bit rate 11 kali. Saluran suara 64 kbps membutuhkan 11 × 64 kbps = 704 kbps. Artinya saluran bandpass dapat membawa (10 Mbps) / (704 kbps) atau sekitar 14 saluran.
 \end{solution}

\end{document}

