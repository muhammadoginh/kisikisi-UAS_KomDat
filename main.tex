\documentclass[oneside]{book}
\usepackage{xcolor}


\newcommand{\exercisename}{Latihan}
\newcommand{\solutionname}{Solusi}

\definecolor{main}{RGB}{0,120,2}

%% Exercise with counter
\newcounter{exer}[chapter]
\setcounter{exer}{0}
\renewcommand{\theexer}{\thechapter.\arabic{exer}}
\newenvironment{exercise}[1][]{
  \refstepcounter{exer}
  \par\noindent\textbf{\color{main}{\exercisename} \theexer #1 }\rmfamily}{
  \par\ignorespacesafterend}

\newenvironment{solution}{\par\noindent\textbf{\color{main}\solutionname} \em}{\par}

\begin{document}

\chapter{Digital Transmission}

\begin{exercise}
  Contoh soal
\end{exercise}

\begin{solution}
  Contoh solusi
\end{solution}

\chapter{Analog Transmission}

\begin{exercise}
  Calculate the baud rate for the given bit rate and type of modulation.
  \begin{itemize}
    \item[a.] 2000 bps, FSK
    \item[b.] 4000 bps, ASK
  \end{itemize}
\end{exercise}

\begin{solution}
  We use the formula $S = (1/r) \times N$, but first we need to calculate the value of r for each case.
  \begin{itemize}
    \item[a.] $r = log_22 = 1 \quad \rightarrow \quad S = (1/1) \times (2000 \textnormal{ bps}) = 2000 \textnormal{ baud}$
    \item[b.] 
  \end{itemize}
\end{solution}

\vspace{12pt}

\begin{exercise}
  Contoh soal
\end{exercise}

\begin{solution}
  Contoh solusi
\end{solution}

\chapter{Bandwidth Utilization: Multiplexing and Spreading}

\begin{exercise}
  Contoh soal
\end{exercise}

\begin{solution}
  Contoh solusi
\end{solution}

\end{document}

